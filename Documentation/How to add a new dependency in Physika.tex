\documentclass[11pt,fullpage]{article}
%\usepackage{multicol,wrapfig,amsmath,subfigure}
\usepackage{multicol,amsmath,amssymb,algorithmic}
\usepackage{graphics,graphicx}
\usepackage[right = 2.5cm, left=2.5cm, top = 2.5cm, bottom =2.5cm]{geometry}
\pagestyle{plain}
\def\urlfont{\DeclareFontFamily{OT1}{cmtt}{\hyphenchar\font='057}
              \normalfont\ttfamily \hyphenpenalty=10000}

%\input macros

\newcommand{\subheading}[1]{\noindent \textbf{#1}}
\newcommand{\grad}{\nabla}
\newcommand{\jump}[1]{[#1]}
\newcommand{\limit}[2]{\lim_{#1 \rightarrow #2}}
\newcommand{\mb}[1]{\mathbf{#1}}
\newcommand{\reals}[1]{\mathbb{R}^{#1}}


\begin{document}
\noindent
\begin{center}
{\bf\large  How to add a new dependency in Physika?}
{\\ Fei Zhu \\ 05/05/2014}
\end{center}

\subheading{Step 1: Is it really necessary?}

\noindent{}We're very cautious about adding a new dependency in
Physika. So please ask yourself the following questions before you add one:
\begin{itemize}
\item{Is the functionality you need already in Physika? We avoid
    adding dependencies that have redundant functionality. If the
    functionality you need is already covered by either Physika itself or
    any existing dependency, don't add a new one.}
\item{Is the functionality you need easy to implement with short code? If so,
    implement it yourself.}
\item{Are you confident that the library to be imported will be used frequently by
    all(most) Physika developers? Generally we don't recommend to import
    3rd code(probably huge) merely for the need of a few.}
\end{itemize}

\subheading{Step 2: Add a new dependency.}

\noindent{}The path of Physika dependency is
``Physika\_{}Src/Physika\_{}Dependency/''. The dependency in Physika may be provided in \textbf{3}
forms:
\begin{enumerate}
\item{\textbf{Only headers.} This is when the 3rd party code is
    written in C++ template. In this case, place the code in
    corresponding dependency path and update the ``pure\_{}copy''
    variable in scons script of Physika. The headers will be copied to
  target include directory while building Physika.}
\item{\textbf{Source code.} In this case, you need to write a scons
    script for the library, rename it as ``SConscript'' and place it
    at root directory of this library. Take the ``LodePNG'' library as
    an example, the script is put in
    ``Physika\_{}Src/Physika\_{}Dependency/LodePNG/''. In this script you need to
    compile the library into libs and copy the headers and libs to
    target build directories of Physika. The build script will be called
  by build script of Physika. I provide a template of this script at
  root directory of dependency.}
\item{\textbf{Headers and precompiled libraries.} In this case, you
    need to put all headers in a subdirectory of the library with the
    name ``include'' and put all libs in a subdirectory called
    ``lib''. The libs are categorized according to the OS name and
    architecture. In ``lib'' directory, there are 3 directories named
    as ``Apple'',``Linux'', and ``Windows''. Each of the 3 directories
  has 2 subdirectories called ``X86'' and ``X64''. The libs for
  different platforms are put in corresponding directories
  respectively. For example, the lib file for 64 bit Windows is put in
``LibraryName/lib/Windows/X64/''. The build script of Physika will be
able to find lib files for the target platform and copy them to public
directories. \emph{Please provide lib files for all supported platforms
while adding a new dependency!}}
\end{enumerate}

\subheading{Step 3: Inform other developers.}

\noindent{}It's a big decision anyway, they deserve to be noted.

\end{document}